%%%%%%%%%%%%%%%%%%%%%%%%%%%%%%%%%%%%%%%%%
% Structured General Purpose Assignment
% LaTeX Template
%
% This template has been downloaded from:
% http://www.latextemplates.com
%
% Original author:
% Ted Pavlic (http://www.tedpavlic.com)
%
% Note:
% The \lipsum[#] commands throughout this template generate dummy text
% to fill the template out. These commands should all be removed when 
% writing assignment content.
%
%%%%%%%%%%%%%%%%%%%%%%%%%%%%%%%%%%%%%%%%%


%----------------------------------------------------------------------------------------
%	PACKAGES AND OTHER DOCUMENT CONFIGURATIONS
%----------------------------------------------------------------------------------------

\documentclass{article}

%\usepackage{currfile}
\usepackage{fancyhdr} % Required for custom headers
\usepackage{lastpage} % Required to determine the last page for the footer
\usepackage{extramarks} % Required for headers and footers
\usepackage{graphicx} % Required to insert images
\usepackage{lipsum} % Used for inserting dummy 'Lorem ipsum' text into the template
\usepackage{outlines}
\usepackage{wrapfig}
\usepackage[dutch,]{babel}

% toegevoegd voor checkmark
\usepackage{tikz}


\selectlanguage{dutch}


% Margins
\topmargin=-0.45in
\evensidemargin=0in
\oddsidemargin=0in
\textwidth=6.5in
\textheight=9.0in
\headsep=0.25in 

\linespread{1.1} % Line spacing

% Set up the header and footer
\pagestyle{fancy}
\lhead{\hmwkAuthorName} % Top left header
%\chead{\hmwkClass\ \small{(\textit{\hmwkClassInstructor})}} 
\chead{\hmwkClass} 
\rhead{\hmwkTitle} 
\lfoot{\LaTeX: \small{\input{filename.txt}}} % Bottom left footer
%\lfoot{\LaTeX: {/home/jan/CVOTSM/A7\_IT-organisatie/ITIL/}\currfilepath} % Bottom left footer
\cfoot{} % Bottom center footer
\rfoot{Pagina\ \thepage\ van\ \pageref{LastPage}} % Bottom right footer
\renewcommand\headrulewidth{0.4pt} % Size of the header rule
\renewcommand\footrulewidth{0.4pt} % Size of the footer rule

\setlength\parindent{0pt} % Removes all indentation from paragraphs


%\setlength{\parskip}{\baselineskip}%
%\setlength{\parindent}{12pt}%

%----------------------------------------------------------------------------------------
%	DOCUMENT STRUCTURE COMMANDS
%	Skip this unless you know what you're doing
%----------------------------------------------------------------------------------------

% Header and footer for when a page split occurs within a problem environment
\newcommand{\enterProblemHeader}[1]{
\nobreak\extramarks{#1}{#1 continued on next page\ldots}\nobreak
\nobreak\extramarks{#1 (continued)}{#1 continued on next page\ldots}\nobreak
}

% Header and footer for when a page split occurs between problem environments
\newcommand{\exitProblemHeader}[1]{
\nobreak\extramarks{#1 (continued)}{#1 continued on next page\ldots}\nobreak
\nobreak\extramarks{#1}{}\nobreak
}

\setcounter{secnumdepth}{0} % Removes default section numbers
\newcounter{homeworkProblemCounter} % Creates a counter to keep track of the number of problems

\newcommand{\homeworkProblemName}{}
\newenvironment{homeworkProblem}[1][Problem \arabic{homeworkProblemCounter}]{ % Makes a new environment called homeworkProblem which takes 1 argument (custom name) but the default is "Problem #"
\stepcounter{homeworkProblemCounter} % Increase counter for number of problems
\renewcommand{\homeworkProblemName}{#1} % Assign \homeworkProblemName the name of the problem
\section{\homeworkProblemName} % Make a section in the document with the custom problem count
\enterProblemHeader{\homeworkProblemName} % Header and footer within the environment
}{
\exitProblemHeader{\homeworkProblemName} % Header and footer after the environment
}

\newcommand{\problemAnswer}[1]{ % Defines the problem answer command with the content as the only argument
\noindent\framebox[\columnwidth][c]{\begin{minipage}{0.98\columnwidth}#1\end{minipage}} % Makes the box around the problem answer and puts the content inside
}

\newcommand{\homeworkSectionName}{}
\newenvironment{homeworkSection}[1]{ % New environment for sections within homework problems, takes 1 argument - the name of the section
\renewcommand{\homeworkSectionName}{#1} % Assign \homeworkSectionName to the name of the section from the environment argument
\subsection{\homeworkSectionName} % Make a subsection with the custom name of the subsection
\enterProblemHeader{\homeworkProblemName\ [\homeworkSectionName]} % Header and footer within the environment
}{
\enterProblemHeader{\homeworkProblemName} % Header and footer after the environment
}
   
%----------------------------------------------------------------------------------------
%	NAME AND CLASS SECTION
%----------------------------------------------------------------------------------------

\newcommand{\hmwkTitle}{3. Feature List} % Assignment title
%\newcommand{\hmwkDueDate}{Monday,\ January\ 1,\ 2012} % Due date
\newcommand{\hmwkDueDate}{} % Due date
\newcommand{\hmwkClass}{Projectwerk} % Course/class
%\newcommand{\hmwkClassTime}{10:30am} % Class/lecture time
\newcommand{\hmwkClassTime}{} % Class/lecture time
\newcommand{\hmwkClassInstructor}{} % Teacher/lecturer
\newcommand{\hmwkAuthorName}{Wagemakers Jan} % Your name

%----------------------------------------------------------------------------------------
%	TITLE PAGE
%----------------------------------------------------------------------------------------

\title{
\vspace{2in}
\textmd{\textbf{\hmwkClass}}\\
\textmd{\textbf{\hmwkTitle}}\\
%\normalsize\vspace{0.1in}\small{In\ te\ dienen\ voor\ \hmwkDueDate}\\
%\vspace{0.1in}{\textit{Leerkracht: \hmwkClassInstructor\ \hmwkClassTime}}
\vspace{3in}
}

\author{\textbf{\hmwkAuthorName}}
\date{\today} % Insert date here if you want it to appear below your name

%----------------------------------------------------------------------------------------

\begin{document}

\maketitle

%----------------------------------------------------------------------------------------
%	TABLE OF CONTENTS
%----------------------------------------------------------------------------------------

%\setcounter{tocdepth}{1} % Uncomment this line if you don't want subsections listed in the ToC

%\newpage
%\tableofcontents
\newpage

%%% Opdracht
\definecolor{jred}{rgb}{0.91, 0.00, 0.00}
\definecolor{jgreen}{rgb}{0.0, 0.51, 0.40}
\def\checkmark{\color{jgreen}{\tikz\fill[scale=0.4](0,.35) -- (.25,0) -- (1,.7) -- (.25,.15) -- cycle;}} 
%\begin{homeworkProblem}[\arabic{homeworkProblemCounter} : Omschrijving opdracht]
\begin{homeworkProblem}[Feature List]
	\begin{center}
		\begin{tabular}{ | l | c | c | c | }	\hline


			\textbf{\color{jred}Feature} 	& 
			\textbf{\color{jred}{\rotatebox[origin=c]{90}{~Bouwsteen~}}} & 
			\textbf{\color{jred}{\rotatebox[origin=c]{90}{~``Must Have''~}}} & 
			\textbf{\color{jred}{\rotatebox[origin=c]{90}{~``Pepper \& salt''~}}} 	
			\\ \hline
		
			Connectie met Database 						&  \checkmark	&     ~		&	~		\\ \hline
		    	Gegevens HoogspanningsTransfo tonen 				&  \checkmark	&     ~		&	~	 	\\ \hline
			Gegevens VerdeelBord tonen					&  \checkmark	&     ~		&	~		\\ \hline
			Gegevens ZekeringKast tonen					&  \checkmark	&     ~		&	~		\\ \hline
			Gegevens HoogspanningsTransfo aanpassen 			&  \checkmark	&     ~		&	~		\\ \hline
			Gegevens VerdeelBord aanpassen					&  \checkmark	&     ~		&	~		\\ \hline
			Gegevens ZekeringKast aanpassen					&  \checkmark	&     ~		&	~		\\ \hline
 		    	Gegevens HoogspanningsTransfo opslaan 				&  \checkmark	&     ~		&	~		\\ \hline
			Gegevens VerdeelBord opslaan					&  \checkmark	&     ~		&	~		\\ \hline
			Gegevens ZekeringKast opslaan					&  \checkmark	&     ~		&	~		\\ \hline

			Gegevens HoogspanningsTransfo afdrukken         		&  ~   		&  \checkmark   &       ~               \\ \hline
                        Gegevens VerdeelBord afdrukken                  		&  ~   		&  \checkmark   &       ~               \\ \hline
			Gegevens ZekeringKast afdrukken                 		&  ~   		&  \checkmark   &       ~               \\ \hline			
			Zoeken op Machine			        		&  ~   		&  \checkmark   &       ~               \\ \hline
			Zoeken op Omschrijving			        		&  ~   		&  \checkmark   &       ~               \\ \hline
			Zoeken op HoogspanningsTransfo					&  ~            &  \checkmark   &       ~               \\ \hline
			Zoeken op Verdeelbord						&  ~            &  \checkmark   &       ~               \\ \hline
			Zoeken op ZekeringKast						&  ~            &  \checkmark   &       ~               \\ \hline
			Afdrukken van zoekresultaten					&  ~            &  \checkmark   &       ~               \\ \hline

			Extra informatie (foto's/elek. schema's) importeren		&  ~   		&     ~         &       \checkmark      \\ \hline
			Extra informatie (foto's/elek. schema's) tonen			&  ~            &     ~         &       \checkmark      \\ \hline
			Extra informatie (foto's/elek. schema's) afdrukken		&  ~            &     ~         &       \checkmark      \\ \hline
			Veiligheidsvoorschriften/wetgeving raadplegen via programma 	&  ~            &     ~         &       \checkmark      \\ \hline
			Totaaloverzicht/Automatische generatie eendraadschema's		&  ~            &     ~         &       \checkmark      \\ \hline
			Delen van informatie (bv. via E-mail)				&  ~            &     ~         &       \checkmark      \\ \hline
			Grondplan + automatische generatie dmv. locatie-gegevens	&  ~            &     ~         &       \checkmark      \\ \hline




		    \end{tabular}
	\end{center}
\end{homeworkProblem}

\end{document}
