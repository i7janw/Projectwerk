%%%%%%%%%%%%%%%%%%%%%%%%%%%%%%%%%%%%%%%%%
% Structured General Purpose Assignment
% LaTeX Template
%
% This template has been downloaded from:
% http://www.latextemplates.com
%
% Original author:
% Ted Pavlic (http://www.tedpavlic.com)
%
% Note:
% The \lipsum[#] commands throughout this template generate dummy text
% to fill the template out. These commands should all be removed when 
% writing assignment content.
%
%%%%%%%%%%%%%%%%%%%%%%%%%%%%%%%%%%%%%%%%%


%----------------------------------------------------------------------------------------
%	PACKAGES AND OTHER DOCUMENT CONFIGURATIONS
%----------------------------------------------------------------------------------------

\documentclass{article}

%\usepackage{currfile}
\usepackage{fancyhdr} % Required for custom headers
\usepackage{lastpage} % Required to determine the last page for the footer
\usepackage{extramarks} % Required for headers and footers
\usepackage{graphicx} % Required to insert images
\usepackage{lipsum} % Used for inserting dummy 'Lorem ipsum' text into the template
\usepackage{outlines}
\usepackage{wrapfig}
\usepackage[dutch,]{babel}
\selectlanguage{dutch}


% Margins
\topmargin=-0.45in
\evensidemargin=0in
\oddsidemargin=0in
\textwidth=6.5in
\textheight=9.0in
\headsep=0.25in 

\linespread{1.1} % Line spacing

% Set up the header and footer
\pagestyle{fancy}
\lhead{\hmwkAuthorName} % Top left header
%\chead{\hmwkClass\ \small{(\textit{\hmwkClassInstructor})}} 
\chead{\hmwkClass} 
\rhead{\hmwkTitle} 
\lfoot{\LaTeX: \small{\input{filename.txt}}} % Bottom left footer
%\lfoot{\LaTeX: {/home/jan/CVOTSM/A7\_IT-organisatie/ITIL/}\currfilepath} % Bottom left footer
\cfoot{} % Bottom center footer
\rfoot{Pagina\ \thepage\ van\ \pageref{LastPage}} % Bottom right footer
\renewcommand\headrulewidth{0.4pt} % Size of the header rule
\renewcommand\footrulewidth{0.4pt} % Size of the footer rule

\setlength\parindent{0pt} % Removes all indentation from paragraphs


%\setlength{\parskip}{\baselineskip}%
%\setlength{\parindent}{12pt}%

%----------------------------------------------------------------------------------------
%	DOCUMENT STRUCTURE COMMANDS
%	Skip this unless you know what you're doing
%----------------------------------------------------------------------------------------

% Header and footer for when a page split occurs within a problem environment
\newcommand{\enterProblemHeader}[1]{
\nobreak\extramarks{#1}{#1 continued on next page\ldots}\nobreak
\nobreak\extramarks{#1 (continued)}{#1 continued on next page\ldots}\nobreak
}

% Header and footer for when a page split occurs between problem environments
\newcommand{\exitProblemHeader}[1]{
\nobreak\extramarks{#1 (continued)}{#1 continued on next page\ldots}\nobreak
\nobreak\extramarks{#1}{}\nobreak
}

\setcounter{secnumdepth}{0} % Removes default section numbers
\newcounter{homeworkProblemCounter} % Creates a counter to keep track of the number of problems

\newcommand{\homeworkProblemName}{}
\newenvironment{homeworkProblem}[1][Problem \arabic{homeworkProblemCounter}]{ % Makes a new environment called homeworkProblem which takes 1 argument (custom name) but the default is "Problem #"
\stepcounter{homeworkProblemCounter} % Increase counter for number of problems
\renewcommand{\homeworkProblemName}{#1} % Assign \homeworkProblemName the name of the problem
\section{\homeworkProblemName} % Make a section in the document with the custom problem count
\enterProblemHeader{\homeworkProblemName} % Header and footer within the environment
}{
\exitProblemHeader{\homeworkProblemName} % Header and footer after the environment
}

\newcommand{\problemAnswer}[1]{ % Defines the problem answer command with the content as the only argument
\noindent\framebox[\columnwidth][c]{\begin{minipage}{0.98\columnwidth}#1\end{minipage}} % Makes the box around the problem answer and puts the content inside
}

\newcommand{\homeworkSectionName}{}
\newenvironment{homeworkSection}[1]{ % New environment for sections within homework problems, takes 1 argument - the name of the section
\renewcommand{\homeworkSectionName}{#1} % Assign \homeworkSectionName to the name of the section from the environment argument
\subsection{\homeworkSectionName} % Make a subsection with the custom name of the subsection
\enterProblemHeader{\homeworkProblemName\ [\homeworkSectionName]} % Header and footer within the environment
}{
\enterProblemHeader{\homeworkProblemName} % Header and footer after the environment
}
   
%----------------------------------------------------------------------------------------
%	NAME AND CLASS SECTION
%----------------------------------------------------------------------------------------

\newcommand{\hmwkTitle}{1. Korte omschrijving} % Assignment title
%\newcommand{\hmwkDueDate}{Monday,\ January\ 1,\ 2012} % Due date
\newcommand{\hmwkDueDate}{} % Due date
\newcommand{\hmwkClass}{Projectwerk} % Course/class
%\newcommand{\hmwkClassTime}{10:30am} % Class/lecture time
\newcommand{\hmwkClassTime}{} % Class/lecture time
\newcommand{\hmwkClassInstructor}{} % Teacher/lecturer
\newcommand{\hmwkAuthorName}{Wagemakers Jan} % Your name

%----------------------------------------------------------------------------------------
%	TITLE PAGE
%----------------------------------------------------------------------------------------

\title{
\vspace{2in}
\textmd{\textbf{\hmwkClass}}\\
\textmd{\textbf{\hmwkTitle}}\\
%\normalsize\vspace{0.1in}\small{In\ te\ dienen\ voor\ \hmwkDueDate}\\
%\vspace{0.1in}{\textit{Leerkracht: \hmwkClassInstructor\ \hmwkClassTime}}
\vspace{3in}
}

\author{\textbf{\hmwkAuthorName}}
\date{\today} % Insert date here if you want it to appear below your name

%----------------------------------------------------------------------------------------

\begin{document}

\maketitle

%----------------------------------------------------------------------------------------
%	TABLE OF CONTENTS
%----------------------------------------------------------------------------------------

%\setcounter{tocdepth}{1} % Uncomment this line if you don't want subsections listed in the ToC

%\newpage
%\tableofcontents
\newpage

%%% Opdracht

%\begin{homeworkProblem}[\arabic{homeworkProblemCounter} : Omschrijving opdracht]
\begin{homeworkProblem}[1. Korte omschrijving]

Het bedrijf \textbf{``Hansen Industrial Transmissions''} heeft een laagspanningsnet
bestaande uit 5 hoogspanningsposten waaraan
machines/verdeelborden/zekeringkasten aangesloten zijn.

Momenteel worden deze gegevens door de onderhoudstechniekers bijgehouden
door middel van verschillende losse Excel-rekenbladen. 
\\\\
Bedoeling van de software is om te beginnen dat hetgene wat men nu doet,
kan blijven doen. Dit houdt in dat men
aanpassingen aan het laagspanningsnet eenvoudig kan ingeven en dat een
overzichtelijk blad kan afgedrukt worden dat in de desbetreffende zekeringkast of
verdeelbord gestoken wordt.
\\\\
Het grootste nadeel van het gebruik van een rekeblad voor het beheer van het
laagspanningsnet is dat het opzoeken van gegevens zeer moeilijk is. Een
eenvoudige vraag zoals \textit{``waar is machine R052 aangesloten''}
is niet eenvoudig te beantwoorden. Het belangrijkste doel van de te
schrijven software is dan ook om zoekfuncties in te bouwen. Men zal een
machine-nummer kunnen ingeven en dan zal men doorverwezen worden naar de
zekeringkast/verdeelbord/hoofdsleutel waar de machine op is aangesloten.
Ook het zoeken op omschrijving moet mogelijk zijn om zo toestellen zonder
nummer terug te vinden. Denk aan een omschrijving zoals ``draaibomen''. In
dat geval zullen er dus meerdere resultaten (lees meerdere zekeringkasten)
gevonden worden die via een duidelijk overzicht aan de gebruiker getoond
wordt.
\\\\
Het idee is om de lijsten van
zekeringkasten/verdeelborden/hoofdsleutels navigeerbaar te maken via
hyperlinks. Een beetje te vergelijken met hoe men over een webpagina op het
internet navigeert. Stel het blad van ``Transfo 8'' is zichtbaar op het
scherm. Als men dan klikt op hoofdsleutel ``H806'' zal men naar het blad van
``VB806'' springen. Klikt men dan op bv. ``vertrek C'' dan zal men bv. naar
het blad van ``K806c'' springen. 
\\\\
In het bedrijf \textbf{``Hansen Industrial Transmissions''} hebben de
onderhoudsmedewerker toegang tot PC's waar het besturingssysteem
\textbf{Windows 7} op ge\"installeerd is. De software zal dus bestaan uit
een desktop-applicatie die op deze PC's ge\"installeerd wordt.
\\\\
Bepaalde zekeringkasten hebben extra funtionaliteit. Denk aan bv. de
zekeringkasten met impulsschakelaars om de verlichting te sturen. Voor
dergelijke kasten kan het wenselijk zijn dat er schema's toegevoegd worden.
Zo ook voor bv. eendraadsschema's van verdeelborden.
Daarom kan het handig zijn dat men deze schema's of andere documenten mee
kan importeren. Het is echter geen top-prioriteit om dit importeren van
documenten te programmeren omdat er momenteel van deze zaken meestal (nog) geen correcte
schema's voorhanden zijn. 
\\\\

  




\end{homeworkProblem}

% ==========================================================================================================================================

\end{document}
