%%%%%%%%%%%%%%%%%%%%%%%%%%%%%%%%%%%%%%%%%
% Structured General Purpose Assignment
% LaTeX Template
%
% This template has been downloaded from:
% http://www.latextemplates.com
%
% Original author:
% Ted Pavlic (http://www.tedpavlic.com)
%
% Note:
% The \lipsum[#] commands throughout this template generate dummy text
% to fill the template out. These commands should all be removed when 
% writing assignment content.
%
%%%%%%%%%%%%%%%%%%%%%%%%%%%%%%%%%%%%%%%%%


%----------------------------------------------------------------------------------------
%	PACKAGES AND OTHER DOCUMENT CONFIGURATIONS
%----------------------------------------------------------------------------------------

\documentclass{article}

%\usepackage{currfile}
\usepackage{fancyhdr} % Required for custom headers
\usepackage{lastpage} % Required to determine the last page for the footer
\usepackage{extramarks} % Required for headers and footers
\usepackage{graphicx} % Required to insert images
\usepackage{lipsum} % Used for inserting dummy 'Lorem ipsum' text into the template
\usepackage{outlines}
\usepackage{wrapfig}
\usepackage[dutch,]{babel}
\selectlanguage{dutch}


% Margins
\topmargin=-0.45in
\evensidemargin=0in
\oddsidemargin=0in
\textwidth=6.5in
\textheight=9.0in
\headsep=0.25in 

\linespread{1.1} % Line spacing

% Set up the header and footer
\pagestyle{fancy}
\lhead{\hmwkAuthorName} % Top left header
%\chead{\hmwkClass\ \small{(\textit{\hmwkClassInstructor})}} 
\chead{\hmwkClass} 
\rhead{\hmwkTitle} 
\lfoot{\LaTeX: \small{\input{filename.txt}}} % Bottom left footer
%\lfoot{\LaTeX: {/home/jan/CVOTSM/A7\_IT-organisatie/ITIL/}\currfilepath} % Bottom left footer
\cfoot{} % Bottom center footer
\rfoot{Pagina\ \thepage\ van\ \pageref{LastPage}} % Bottom right footer
\renewcommand\headrulewidth{0.4pt} % Size of the header rule
\renewcommand\footrulewidth{0.4pt} % Size of the footer rule

\setlength\parindent{0pt} % Removes all indentation from paragraphs


%\setlength{\parskip}{\baselineskip}%
%\setlength{\parindent}{12pt}%

%----------------------------------------------------------------------------------------
%	DOCUMENT STRUCTURE COMMANDS
%	Skip this unless you know what you're doing
%----------------------------------------------------------------------------------------

% Header and footer for when a page split occurs within a problem environment
\newcommand{\enterProblemHeader}[1]{
\nobreak\extramarks{#1}{#1 continued on next page\ldots}\nobreak
\nobreak\extramarks{#1 (continued)}{#1 continued on next page\ldots}\nobreak
}

% Header and footer for when a page split occurs between problem environments
\newcommand{\exitProblemHeader}[1]{
\nobreak\extramarks{#1 (continued)}{#1 continued on next page\ldots}\nobreak
\nobreak\extramarks{#1}{}\nobreak
}

\setcounter{secnumdepth}{0} % Removes default section numbers
\newcounter{homeworkProblemCounter} % Creates a counter to keep track of the number of problems

\newcommand{\homeworkProblemName}{}
\newenvironment{homeworkProblem}[1][Problem \arabic{homeworkProblemCounter}]{ % Makes a new environment called homeworkProblem which takes 1 argument (custom name) but the default is "Problem #"
\stepcounter{homeworkProblemCounter} % Increase counter for number of problems
\renewcommand{\homeworkProblemName}{#1} % Assign \homeworkProblemName the name of the problem
\section{\homeworkProblemName} % Make a section in the document with the custom problem count
\enterProblemHeader{\homeworkProblemName} % Header and footer within the environment
}{
\exitProblemHeader{\homeworkProblemName} % Header and footer after the environment
}

\newcommand{\problemAnswer}[1]{ % Defines the problem answer command with the content as the only argument
\noindent\framebox[\columnwidth][c]{\begin{minipage}{0.98\columnwidth}#1\end{minipage}} % Makes the box around the problem answer and puts the content inside
}

\newcommand{\homeworkSectionName}{}
\newenvironment{homeworkSection}[1]{ % New environment for sections within homework problems, takes 1 argument - the name of the section
\renewcommand{\homeworkSectionName}{#1} % Assign \homeworkSectionName to the name of the section from the environment argument
\subsection{\homeworkSectionName} % Make a subsection with the custom name of the subsection
\enterProblemHeader{\homeworkProblemName\ [\homeworkSectionName]} % Header and footer within the environment
}{
\enterProblemHeader{\homeworkProblemName} % Header and footer after the environment
}
   
%----------------------------------------------------------------------------------------
%	NAME AND CLASS SECTION
%----------------------------------------------------------------------------------------

\newcommand{\hmwkTitle}{13. Logboek scrum/agile} % Assignment title
%\newcommand{\hmwkDueDate}{Monday,\ January\ 1,\ 2012} % Due date
\newcommand{\hmwkDueDate}{} % Due date
\newcommand{\hmwkClass}{Projectwerk} % Course/class
%\newcommand{\hmwkClassTime}{10:30am} % Class/lecture time
\newcommand{\hmwkClassTime}{} % Class/lecture time
\newcommand{\hmwkClassInstructor}{} % Teacher/lecturer
\newcommand{\hmwkAuthorName}{Wagemakers Jan} % Your name

%----------------------------------------------------------------------------------------
%	TITLE PAGE
%----------------------------------------------------------------------------------------

\title{
\vspace{2in}
\textmd{\textbf{\hmwkClass}}\\
\textmd{\textbf{\hmwkTitle}}\\
%\normalsize\vspace{0.1in}\small{In\ te\ dienen\ voor\ \hmwkDueDate}\\
%\vspace{0.1in}{\textit{Leerkracht: \hmwkClassInstructor\ \hmwkClassTime}}
\vspace{3in}
}

\author{\textbf{\hmwkAuthorName}}
\date{\today} % Insert date here if you want it to appear below your name

%----------------------------------------------------------------------------------------

\begin{document}

\maketitle

%----------------------------------------------------------------------------------------
%	TABLE OF CONTENTS
%----------------------------------------------------------------------------------------

%\setcounter{tocdepth}{1} % Uncomment this line if you don't want subsections listed in the ToC

%\newpage
%\tableofcontents
\newpage

%%% Opdracht

%\begin{homeworkProblem}[\arabic{homeworkProblemCounter} : Omschrijving opdracht]
\begin{homeworkProblem}[29/10/2017]
We gaan gewoon starten met het schrijven van korte omschrijving
designdocument om zo al een duidelijker beeld te krijgen van wat de software
juist moet doen.
\\~\\
Documenten moeten nog niet 100\% in orde zijn. Belangrijker is dat er een
begin is en er over het probleem goed wordt nagedacht.
\end{homeworkProblem}

\begin{homeworkProblem}[04/11/2017]
Korte omschrijving + definitie-studie : 1ste draft is gemaakt, later nog
terug bekijken en verder uitwerken.
\\~\\
Besloten om als volgende stap de opbouw van de database te bekijken. Dit
hoort bij analyse/datawoordenboek. Reden om hier mee te starten is dat de
database in dit project een zeer belangrijke rol speelt. 
\\~\\
Jan gaat database structuur bekijken en eventueel al eens uittesten in bv.
mySQL workbench.
\end{homeworkProblem}

\begin{homeworkProblem}[11/11/2017]
	Op scrumbord een extra veld \textbf{DRAFT's ter nazicht} aangemaakt om een onderscheid te maken tussen zaken waar als iets voor gedaan is.
\\~\\
	Database : zekeringkasten, verdeelborden, tranfso's worden bezien als \textbf{Aansluitpunten}, van waaruit \textbf{Aansluitingen} naar andere \textbf{Aansluitpunten} of \textbf{Machines} of andere eindverbruikers vertrekken.
	Zie \textbf{datawoordenboek} voor meer info.
	\\~\\
	Tegen volgende week moet er een eerste aanzet gemaakt zijn voor de feature list. 

\end{homeworkProblem}

\begin{homeworkProblem}[18/11/2017]
Feature list is aangemaakt.
\\~\\
Use case diagram + visueel ontwerp gelijktijdig uitwerken omdat er een groot verband is tussen de \textbf{gebruikersschermen} en de \textbf{use cases}.
\end{homeworkProblem}

\begin{homeworkProblem}[26/11/2017]
Feedback doorgenomen.
\\~\\
Aan het datawoordenboek zijn nog een aantal zaken die verbeterd kunnen
worden. Is het bv. een goed idee om met waarden NULL te werken als een
aansluiting niet naar een ander aansluitpunt gaat. Zie ook naar de \textbf{checklist} die bij datawoordenboek toegevoegd is aan het scrum-board.
\\~\\
Een ander aandachtspunt in verband met het datawoordenboek is de vraag of we
de aanduiding van een aansluitpunt als primary key nemen, of toch beter met
een extra ID werken. Feedback aan Joris gevraagd die meer ervaring in deze
materie heeft.
\\~\\
%%\problemAnswer{
Momenteel wordt er dus aan het \textbf{datawoordenboek}, \textbf{gebruikersschermen (visueel
ontwerp)} en \textbf{use cases} gewerkt.
%%}

\end{homeworkProblem}

\begin{homeworkProblem}[30/12/2017]
Door de drukke agenda van het ontwikkelteam zijn de scrum-meetings
uitgesteld geweest.
\\\\
Het datawoordenboek is aangepast om de benamingen consistent'er te maken.
Een SQL-script is aan het
datawoordenboek toegevoegd zo dat hiermee de nodige testen gedaan kunnen
worden bij het ontwikkelen.      
Waarden NULL in datawoordenboek zijn behouden. Een aantal testen hebben
uitgewezen dat dit geen probleem is om op te vangen en het geeft de
mogelijkheid om CONSTRAINT's (zie mySQL script) aan de database toe te voegen.
\\\\
Aan de feature-list zijn nog een aantal zaken toegevoegd.
\\\\
Use case diagram is aangemaakt.
\\\\
Visueel ontwerp is aangemaakt.
\\\\
Volgende stap is om de \textbf{klassendiagrammen} te maken.

\end{homeworkProblem}

\begin{homeworkProblem}[28/01/2018]
Van de \textbf{klassendiagrammen} zijn de verschillende klassen
opgeschreven. De klassen voor communicatie met de database (Database) en het
tonen van het hoofdscherm (Hoofdscherm) zijn op papier (klad) redelijk
uitgewerkt. De andere klassen moeten nog uitgewerkt worden. Er is besloten
om met de verdere uitwerking van het klassendiagram te wachten en te starten
met het programmeren van de database en hoofdscherm klassen aan de hand van
de documentatie die nu is uitgewerkt. Dit om te testen of de theorie die er
is uitgewerkt ook praktisch haalbaar is, of er toch iets bij gestuurd moet
worden.
\\\\
Van het visueel ontwerp moeten er een aantal zaken herbekeken worden:
\begin{itemize}
\item Er zijn knoppen voorzien om naar beneden te bladeren. Bv. Van VB810
naar K810a. Er is echter geen knop voorzien om terug naar boven te gaan. Het
veld voeding op het hoofdscherm kan gebruikt worden als terug-knop.
Aangeduid op de afdruk van het visueel ontwerp. 
\item Het overzicht van de transfo's is een uitzondering ten op zichte van
het tonen van de gegevens van een aansluitpunt. Dit komt in het visueel
ontwerp niet naar voren. Zaken als de +-A knoppen te bekijken.
\item Als er nog zaken naar boven komen, worden deze aangeduid op de afdruk
van het visueel ontwerp.   
\end{itemize}
Vraag use case diagram is nog te bekijken (zie smartschool).
\\\\
Bedoeling is om tegen volgende week voldoende code geschreven te hebben die
laat zien dat de gegevens uit de database gelezen kan worden en er tussen de
verschillende aansluitpunten gebladerd kan worden. Deze broncode uploaden
naar smartschool.
\end{homeworkProblem}

\begin{homeworkProblem}[04/02/2018]
Tijdens het programmeren van de \textbf{Database} en \textbf{Hoofdscherm}
klassen is gebleken dat louter de grafische uitwerking van het
klassendiagram met methodes niet handig is om mee te werken. Daarom zijn de
methodes in een klassendiagram-werkdocument duidelijker omschreven door het
\textbf{doel} van de te programmeren methodes te omschrijven.
\\\\
De \textbf{Database} en \textbf{Hoofdscherm} klassen zijn geschreven in die
mate dat er gebladerd kan worden tussen de verschillende aansluitpunten. Er
kan ook een lijn gewist worden. Bewaren en zoeken werkt ook. Deze code is
echter nog in een beginstadium en met heel wat onvolkomenheden (bv. in de
Database klasse geen rekening houden met de return-value van open() en
close()). Dit moet nog beter afgewerkt worden. Toch laat de huidige code de
mogelijkheden reeds zien.
\\\\
Over zo'n twee weken zouden de attributen en methodes voor de klasse
\textbf{AansluitingAanpassen} uitgewerkt moeten zijn en al een gedeelte
geprogrammeerd moeten zijn. Door deze klasse uit te werken kan dan in de
klasse \textbf{Hoofdscherm} de $+$ en $A$ knop (in de methode
\textbf{dgvLaagspanningsnet\_CellContentClick} geprogrammeerd worden.

\end{homeworkProblem}

\begin{homeworkProblem}[17/02/2018]
Atrributen en methodes voor de klasse \textbf{AansluitingAanpassen} zijn
uitgewerkt en ook geprogrammeerd. In het \textbf{Hoofdscherm} werken de $+$
en $A$ knoppen. Er zijn tijdens het programmeren wel zaken aangepast aan het
klassendiagram. Zo was er in eerste instantie enkel voorzien om een
\textbf{DataRow}
door te geven naar \textbf{AansluitingAanpassen}, maar om de uniekheid van
een aansluiting te testen was het nodig om de volledige \textbf{DataTable} door te
geven.
\\\\
De \textbf{Database} klasse is opgekuisd en er wordt nu wel rekening
gehouden met de return-value van Open() en Close(). Er is echter nog werk
aan, denk bv. aan \textbf{SQL-injection}.
\\\\
Verder is de methode \textbf{GetTransfos()} in de \textbf{Database} klasse
zo aangepast dat deze de lijst van de transfos teruggeeft als een DataSet
van aansluitingen van het aansluitpunt \textbf{Hoogspanning}. Op deze manier
kon de code van de klasse \textbf{Hoofdscherm} sterk vereenvoudigd worden.
\\\\
Opgepast, in tegenstelling tot JAVA beginnen bij C\# de methodes met een
hoofdletter. Aangepast.
\\\\
Volgende stap is om het \textbf{Klassendiagram} uit te werken voor de
klassen Machines/Aansluitpunten - toevoegen/verwijderen/aanpassen. Dit
zou ergens begin maart uitgewerkt moeten zijn. 
Aan alles wat al geprogrammeerd is, is ook nog veel opkuiswerk.
\end{homeworkProblem}

\begin{homeworkProblem}[03/03/2018]
Klassediagram voor de klassen Machines/Aansluitpunten -
toevoegen/verwijderen/aanpassen is gemaakt. De klasse zijn geprogrammeerd en
werken. Dit alles staat nu in testfase, waarbij alles grondig getest moet
worden. Niet alleen de correcte werking moet getest worden, maar ook het
gebruiksgemak. Tijdens het testen alles noteren, om dit later in het
programma mee te nemen.
\\\\
Er was ook reeds begonnen (in klad) aan de klasse voor het afdrukken. De
klasse voor het afdrukken is ook al voor een stuk geprogrammeerd, maar hier
loopt men een beetje vast. Het idee in het klassediagram was om aan de
klasse afdrukken de \textbf{DataGridView} door te geven en deze dan om te zetten naar
iets afdrukbaar. Probleem is dat als we dan \textbf{inclusief
aansluitpunten} willen afdrukken, we deze \textbf{DataGridView} moeten
aanpassen en deze routines in het hoofdscherm zitten. Hier is nu rond
geprogrammeerd, $Hoofdscherm \mapsto Afdrukken \mapsto Hoofdscherm$, maar
analyse/klasse hoofdscherm/afdrukken moet grondig herbekeken worden.
\\\\
De \textbf{Database} klasse verder afwerken.
\\\\
Eind deze maand zou het afdrukken volledig uitgewerkt moeten zijn.
\end{homeworkProblem}

\begin{homeworkProblem}[02/04/2018]
Afdrukken is uitgewerkt. Er is hiervoor een nieuwe klasse
\textbf{LaagSpanningGridView} bijgemaakt.
\\\\
Alle klassen staan nu in testfase. Dit wil zeggen dat zowel de werking, maar ook
de opbouw van de broncode bekeken wordt. Elke opmerking tijdens het
testen wordt opgeschreven zodat er over twee weken een lijst is waar aan
gewerkt kan worden.
\end{homeworkProblem}

\begin{homeworkProblem}[08/05/2018]
Het klassendiagram is aangemaakt van uit het programma en staat klaar ter
nazicht. De kans is groot dat het klassendiagram nog aangepast moet worden
als er aanpassingen aan het programma gemaakt worden. 
\\\\
Verschillende opmerkingen over de werking van het programma zijn genoteerd.
Alle klassen staan nu terug in \textbf{mee bezig}, zodat we \'e\'en voor
\'e\'en
deze opmerkingen kunnen wegwerken.

\end{homeworkProblem}




\end{document}
